\chapter{\LaTeX\ quality assurance}

Before handing your work over to others, there are a few things you can do as a basic quality assurance (QA):
\begin{itemize}
  \item Check your output for missing references (indicated by a double question mark, you may use \verb|pdftotext| to automate this) or look at the blg and log files, where these will be indicated as well
  \item Use a spelling checker
  \begin{itemize}
    \item Overleaf checks your spelling by default.
    \item In \TeX{}studio you can set the spelling check at:\\
      \TeX{}studio Options \(\to\) Configure \TeX{}studio\ldots \(\to\) Language Checking.
    \item You can use Aspell (\url{http://aspell.net/}) to check the spelling in *.tex files from the command line.
    \item If all else fails, you can open a tex file in MS Word and use their spelling and grammar checker
  \end{itemize}
  \item Check consistency of capitalisation in the table of contents
  \item Consistency in bibliography
  \begin{itemize}
    \item Capitalisation of titles
    \item Notation of author names
    \item Abbreviation and spelling of journals
  \end{itemize}
  \item Search through your code for words you have likely misspelled (or mix British and American spelling)
  \begin{itemize}
    \item In Windows: \verb"ls -R *.tex | sls 'teh'"
    \item In Linux: \verb|grep -iR 'teh' *.tex|
  \end{itemize}
  \item Both of these can search for regular expressions (regex), e.g.:
  \begin{itemize}
    \item \verb|"\s[\"']\w"| (grep) or \verb|'\s[''"]\w'| (sls): A space, newline, etc. followed by closing single or double quotes, followed by a word character (e.g. a letter). Most likely, these should be replaced by opening quotes (` or ``).
    \item \verb|"\b(\w+) \1\b"| repeated word
    \item Word starting with two capitals:
    \begin{itemize}
      \item In Windows:\\\verb'ls *.bib | sls -CaseSensitive "\b[A-Z][A-Z]+[a-z]"'
      \item In Linux: \verb|grep -PR "\b[A-Z][A-Z]+[a-z]"|
    \end{itemize}
    \item In your *.bib files:
    \begin{itemize}
      \item \verb|"author\s*=.*\w\.\w"| initial without following space.
      \item \verb|"author\s*=.*\b\w\s\w"| initial without following period.
      \item \verb|"url\s*=.*doi"| DOI given as URL instead of DOI directly
      \item \verb|"doi\s*=.*doi\.org"| DOI including `doi.org'
      \item \verb|"[Uu]rl.*[^\\]#"| Forgot to escape \# in URL
    \end{itemize}
    \item Depending on your citation style:
    \begin{itemize}
      \item space before \verb|cite|
      \item no space before \verb|cite|
      \item parenthesis around \verb|cite|
      \item no parenthesis around \verb|cite|
      \item \ldots
    \end{itemize}
    \item \verb|grep -rnPiR "[^\x09\x0D\x20-\x7E\xBF-\xFF]" *.tex| (or\break\verb|*.bib| `weird' characters (that may look normal) probably caused by copying something from the a pdf file, which might cause \LaTeX\ to error.
    \item When not using 'french spacing': patterns which likely confuse periods indicating an abbreviation with end of sentence and vice versa.
  \end{itemize}
  \item Check your output for missing references (indicated by a double question mark, you may use \verb|pdftotext| to automate this) or look at the log files, where these will be indicated as well
  \item Chk\TeX\ (\url{https://www.ctan.org/pkg/chktex}) is written specifically to search for common errors in *.tex files
  \item All of these suggestions work best when automated, for instance in a Makefile. Doing it by hand is way too boring, so you are very likely to forget running some of these tests on the final version of your thesis.
\end{itemize}
