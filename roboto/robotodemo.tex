\section{Roboto}

The TU Delft style prescribes Roboto Slab together with Arial.%
\footnote{\url{https://www.tudelft.nl/huisstijl/bouwstenen/typografie}}
Since Arial is not available in pdf\LaTeX, we will use regular Roboto instead.
Not surprisingly, this looks great together with Roboto Slab.

Roboto (Sans) has a dual nature. It has a mechanical skeleton
and the forms are largely geometric. At the same time,
the font features friendly and open curves. While some
grotesks distort their letterforms to force a rigid
rhythm, Roboto doesn't compromise, allowing letters to be
settled into their natural width. This makes for a more
natural reading rhythm more commonly found in humanist and
serif types.

Roboto Serif is designed to create a comfortable
and frictionless reading experience. Minimal and
highly functional, it is useful anywhere (even for app
interfaces) due to the extensive set of weights and widths
across a broad range of optical sizes. While it was
carefully crafted to work well in digital media, across
the full scope of sizes and resolutions we have today, it
is just as comfortable to read and work in print media.

Roboto has several styles of digits:
\begin{itemize}
    \item `Normal' lining numbers
    \begin{itemize}
        \item Proportional: \robotoLF{1234567890}
        \item Tabular: \robotoTLF{1234567890}
    \end{itemize}
    \item Ols style numbers
    \begin{itemize}
        \item Proportional: \robotoOsF{1234567890}
        \item Tabular: \robotoTOsF{1234567890}
    \end{itemize}
\end{itemize}

Furthermore, the font is available in many different weights:
\begin{itemize}
    \itemsep 0pt
    \parskip 0pt
    \item \robotoThin{robotoThin}
    \item \robotoLight{robotoLight}
    \item \robotoRegular{robotoRegular}     
    \item \robotoMedium{robotoMedium}      
    \item \robotoBold{robotoBold}
    \item \robotoBlack{robotoBlack}    
\end{itemize}

The documentation for the \LaTeX\ package can be found on ctan\footnote{\url{https://ctan.org/pkg/roboto}}; 
The original truetype fonts are available at google\footnote{\url{http://www.google.com/webfonts}}
and are licensed under the
Apache or OFL licenses; the texts may be found in the
doc directory. The opentype and type1 versions were created
using fontforge and cfftot1. The support files were created
using autoinst and are licensed under the terms of the LaTeX
Project Public License. The maintainer of this package is
Bob Tennent (rdt at cs.queensu.ca)
